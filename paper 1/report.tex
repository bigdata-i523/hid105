\documentclass[sigconf]{acmart}

\usepackage{hyperref}

\usepackage{endfloat}
\renewcommand{\efloatseparator}{\mbox{}} % no new page between figures

\usepackage{booktabs} % For formal tables

\settopmatter{printacmref=false} % Removes citation information below abstract
\renewcommand\footnotetextcopyrightpermission[1]{} % removes footnote with conference information in first column
\pagestyle{plain} % removes running headers

\begin{document}
\title{Big Data in Sports Visualization}


\author{Josh Lipe-Melton}
\orcid{1234-5678-9012}
\affiliation{%
  \institution{Indiana University}
  \streetaddress{4400 E Sheffield Dr}
  \city{Bloomington} 
  \state{Indiana} 
  \postcode{47408}
}
\email{jlipemel@umail.iu.edu}

% The default list of authors is too long for headers}
\renewcommand{\shortauthors}{J. Lipe-Melton}


\begin{abstract}
This paper will focus on Big Data Analytics in "other sports" (not baseball) and how it is being used to improve and evaluate performance via spatial visualization
\end{abstract}

\keywords{sports, analytics, data visualization, spatial visualization}


\maketitle

\section{Introduction}

The \textit{proceedings} are the records of a
conference. ACM seeks to give these
conference by-products a uniform, high-quality appearance.  To do
this, ACM has some rigid requirements for the format of the
proceedings documents: there is a specified format (balanced double
columns), a specified set of fonts (Arial or Helvetica and Times
Roman) in certain specified sizes, a specified live area, centered on
the page, specified size of margins, specified column width and gutter
size.

\section{Sports Visualization Use Cases}


\section{Conclusions}

This paragraph will end the body of this sample document.  Remember
that you might still have Acknowledgments or Appendices; brief samples
of these follow.  There is still the Bibliography to deal with; and we
will make a disclaimer about that here: with the exception of the
reference to the \LaTeX\ book, the citations in this paper are to
articles which have nothing to do with the present subject and are
used as examples only.



\end{document}
